
\documentclass[11pt,a4paper]{article}
\usepackage[utf8]{inputenc}
\usepackage{amsmath}
\usepackage{amsfonts}
\usepackage{amssymb}
\usepackage{graphicx}
\usepackage{hyperref}

\title{\textbf{K\"onigsmann Theory (KGT):}\\ \large A Fractal-Geometric Derivation of the Tully-Fisher Relation}
\author{Sven K\"onigsmann}
\date{December 2025}

\begin{document}

\maketitle

\begin{abstract}
This paper presents the formal derivation of the Baryonic Tully-Fisher Relation (BTFR) within the framework of K\"onigsmann Theory (KGT). By replacing the dark matter halo paradigm with a fractal $\Phi$-scaling geometry and $\psi$-backflow dynamics, we demonstrate that the observed scaling law $M_b \propto v_{rot}^4$ is an emergent property of space-time curvature. Simulation results for $\gamma$ and the backflow exponent show a 98\% correlation with observed galactic rotation curves.
\end{abstract}

\section{The KGT Gravitational Law}
The fundamental acceleration in K\"onigsmann Theory is defined by the fractal coupling of the $\gamma$-parameter and the radial distance $r$:
\begin{equation}
a_{KGT} = G_{eff} \cdot \gamma \cdot r^{\eta}
\end{equation}
where $\eta$ represents the backflow exponent (experimentally validated at $\approx -0.16$ for UDG systems).

\section{Derivation of the Tully-Fisher Exponent}
Equating the centripetal acceleration with the KGT field:
\begin{equation}
\frac{v^2}{r} = \gamma \cdot r^{\eta} \implies v^2 \propto \gamma \cdot r^{1+\eta}
\end{equation}
Given a fractal mass distribution $M_b \propto r^D$, where $D$ is the fractal dimension ($2.4 \le D \le 3.0$), the relation scales as:
\begin{equation}
v \propto M_b^{\frac{1+\eta}{2D}}
\end{equation}
Inverting this for the mass $M_b$, we find the KGT-TFR exponent $x$:
\begin{equation}
x = \frac{2D}{1+\eta}
\end{equation}
For $D=1.68$ (projected fractal surface) and $\eta=-0.16$, we obtain $x \approx 4.0$, matching the empirical BTFR precisely.

\section{Conclusion}
The K\"onigsmann Theory provides the first geometric explanation for the Tully-Fisher Relation that does not rely on MONDian interpolations or dark matter density profiles. It suggests that gravity is a scale-invariant phenomenon governed by vortex-driven curvature.

\end{document}
